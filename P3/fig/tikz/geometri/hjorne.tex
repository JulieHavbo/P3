\begin{figure}[h!]
  \centering
  \begin{tikzpicture}[scale=0.8]
    \tikzset{punkt/.style={point, draw=black}}
%
%
% Koordinater
% -------------------------------------------------------
\coordinate (y) at (1,4);
\coordinate (x) at (3,3);
\coordinate (z) at (5,2);
\coordinate (v) at (-3,4);
\coordinate (w) at (-3,2);
\coordinate (u) at (-3,0);
%    
% Punkter
% -------------------------------------------------------
	\node at (3,3)	(1){};
	\node at (-3,3)	(4){};
	\node at (3,-1)	(2){};
	\node at (-3,-1) (3){};
% 
% Firkant 
% -------------------------------------------------------
	\filldraw[mygrey, fill=myblue!30] (2) rectangle (4);
	\node at (0,1) (P){$\mathcal{P}$};
%	
% Punkter og tilhørende tekst
% -------------------------------------------------------
	\filldraw [black] (5,2) circle (0pt) node[right] {$ \{ \mathbf{y} \mid \mathbf{c}_\textbf{w}^T \mathbf{y} = \mathbf{c}_\textbf{w}^T \mathbf{x} \}$};
	\filldraw [black] (x) circle (2pt) node[above] {$\mathbf{x}$};
	\filldraw [black] (-3,3.8) circle (0pt) node[left] {$ \{ \mathbf{y} \mid \mathbf{c}_\textbf{x}^T \mathbf{y} = \mathbf{c}_\textbf{x}^T \mathbf{w} \}$};
	\filldraw [black] (w) circle (2pt) node[left] {$\mathbf{w}$};
%
% Streger mellem punkterne 
% -------------------------------------------------------
	\draw[-,black, thick] (v) -- (w) -- (u);
	\draw[-,black, thick] (y) -- (x);
	\draw[-,black, thick] (x) -- (z);
	\draw[->,black, thick] (-3,1.5) -- (-4.3,1.5) node[left]  {$\mathbf{a}_w$};
	\draw[->,black, thick] (4,2.5) -- (4.5,3.5) node[above] {$\mathbf{a}_x$};
%
\filldraw [black] (-9,2) circle (0pt);
%
  \end{tikzpicture}
  \caption{Et polyeder $\mathcal{P}$, hvor $\textbf{x}$ er et hjørnepunkt, da det er det eneste punkt, som både er i hyperplanet og $\mathcal{P}$.
Vektoren $\textbf{w}$ er derimod ikke et hjørnepunkt, da hyperplanet skærer $\mathcal{P}$ i flere punkter end $\mathbf{w}$.}
  \label{fig:julieermegaseeeeeeeeeej}
\end{figure}
%