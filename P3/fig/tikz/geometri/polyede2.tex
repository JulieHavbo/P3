\begin{minipage}[b]{0.45\textwidth}
%%%%%%%%%%%%%%%%%%%%%%%%%%%%%%%%
%%% Flot graf alla Julie     %%%
%%%%%%%%%%%%%%%%%%%%%%%%%%%%%%%%
%
\begin{center}
%
\begin{tikzpicture}[scale=5]
%
% Koordinater
% -------------------------------------------------------
\coordinate (a) at (1.2,1.2,0.7);
\coordinate (b) at (0.7,1.2,0.7);
\coordinate (c) at (0.7,1.2,1.2);
\coordinate (d) at (1.2,1.2,1.2);
\coordinate (f) at (1.2,0.7,0.7);
\coordinate (g) at (1.2,0.7,1.2);
\coordinate (h) at (0.7,0.7,1.2);
%
% Farvning
% -------------------------------------------------------
\filldraw[fill=myblue,opacity=0.3, thick](c)--(d)--(g)--(h)--(c);
  \draw[thick](d)--(g);
  \draw[thick](d)--(c);
  \draw[thick](g)--(h);
  \draw[thick](h)--(c);
%
% Navngivning og prik til at gøre det pænt 
% -------------------------------------------------------
\draw[black] (0.91,0.96,1.2) circle (0pt) node[anchor=west] {$\mathcal{P}_1$};
\draw[black] (0,0,0.7) circle (0pt);
% 
% 
% Koordinatssystem 
% -------------------------------------------------------
\draw[thick,->] (0,0,0) -- (1,0,0) node[anchor=south east]{$x$};
\draw[thick] (0,0,0) -- (-0.2,0,0);
\draw[thick,->] (0,0,0) -- (0,1,0) node[anchor=north west]{$y$};
\draw[thick] (0,0,0) -- (0,-0.2,0);
%
%
\end{tikzpicture}
  \captionof{figure}{Et polyeder $\mathcal{P}_1 \in \R^2$ markeret med blå.}
  \label{fig:nej2}
\end{center}
%
\end{minipage}