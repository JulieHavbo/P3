%%%%%%%%%%%%%%%%%%%%%%
%%                  %%
%%      MASTER      %%
%%                  %%
%%%%%%%%%%%%%%%%%%%%%%

% master.tex : master-fil for projektet

% --------------------------------------------------

% Dette er hovedfilen for projektet, 
% hvori indhold fra alle input-filer 
% (tekst, billeder, litteraturdatabaser, osv.) samles

% Dokumenttypen 'book' er valgt pga. dens mange fleksible indstillinger
% Se https://tex.stackexchange.com/a/36989/118167

\documentclass[11pt,a4paper,oneside,openright,danish]{book}

% Variabler, som bruges til automatisk at indsætte titel, forfattere, osv. på forsiden og titelbladet.

\def \projecttitle       {Emne}
\def \projectsubtitle    {Titel}
\def \projecttheme       {Tema}
\def \projectdegree      {Matematik}
\def \projectperiod      {Semester og år}
\def \projectnumber      {Ptal}
\def \projectgroup       {Lokale}
\def \projectauthors     {
	Jens Koldkur				\\
	Jonas Bach Christensen		\\
	Julie Havbo Lund			\\
	Mads Bjerregaard Kjær		\\
	Mathias Gammelgaard			\\
	Rasmus Jespersgaard
}
\def \projectsupervisors {
  Vejleder
}

% Preamblet indeholder alle de indstillinger og makroer, som skal indsættes for hovedindholdet, og i denne skabelon samles det i filen aaumath.sty, som definerer en pakke, der kan indlæses med \usepackage.

\usepackage{aaumath}

% Dokumentets indhold indsættes mellem \begin- og \end-makroerne for 'document'-blokken

\begin{document}

% Dokumentets 'front matter' tælles ikke med ifm. antal sider og nummereres med
% romerske tal. Herunder hører f.eks. forsiden, titelbladet, forordet og
% indholdsfortegnelsen.

\frontmatter
\include{incl/misc/frontpage}
\include{incl/misc/titlepage}
\include{incl/misc/contents}

% Dokumentets 'main matter' (hovedindhold) er der, hvor det meste indhold skal sættes ind. Sider og overskrifter er nummererede med arabiske tal.

\mainmatter

% Input-filer bør opdeles således, at hver fil svarer til et kapitel. 
% Makroen \include indsætter et sideskift og indholdet fra den givne stil.

%%%%%%%%%%%%%%%%%%%%%%
%%     INCLUDE      %%
%%%%%%%%%%%%%%%%%%%%%%

% INDLEDNING
	\chapter*{Forord}
\addcontentsline{toc}{chapter}{\phantom{lol.}Forord}
% God arbejdslyst xD
 


\section*{Underskrifter}
\begin{center}
\begin{minipage}[b]{0.45\textwidth}
\begin{center}
\begin{tabular}{l}
\phantom{Julie er virkelig dejlig, sød og smuk} \\
\\
\\
\hline
Jens Koldkur \\
\\
\\
\\
\\
\hline
Jonas Bach Christensen \\
\\
\\
\\
\\
\hline
Julie Havbo Lund \\         
\end{tabular}
\end{center}
\end{minipage}
%
\begin{minipage}[b]{0.038\textwidth}
\phantom{xD}
\end{minipage}
\begin{minipage}[b]{0.45\textwidth}
\begin{center}
\begin{tabular}{l}
\phantom{Julieervirkeligdejligogsødogsmukogs} \\
\\
\\
\hline
Mads Bjerregaard Kjær \\
\\
\\
\\
\\
\hline
Mathias Gammelgaard \\
\\
\\
\\
\\
\hline
Rasmus Jespersgaard \\         
\end{tabular}
\end{center}
\end{minipage}
\end{center}
	\chapter*{Indledning}
\addcontentsline{toc}{chapter}{\phantom{lol.}Indledning}
% 
% Noter:
% To til tre sider 
% Vi skal argumentere det er interessant at kigge på lineær programmering
% Kort beskrivelse af hvad vi har lavet 
% Eventuelt 
%   Kapitel 1: bla bla 
%   Kapitel 2: bla bla 
%   m.m.

%
\begin{col}{}{}
Problemformulering
\end{col}
% 
\noindent
%
\\\\
Nedenstående er en oversigt over indholdet af nærværende rapport.
%
\begin{itemize}[itemindent=0em]
\item[]\textbf{Kapitel 1} l
\item[]\textbf{Kapitel 2} l
\item[]\textbf{Kapitel 3} l
\item[]\textbf{Kapitel 4} l
\end{itemize}
%
	
% TEORI
	\include{incl/main/tema/teori}
	
% Konklusion og perspektivering 
	\chapter{Konklusion}
% God arbejdslyst xD	
	%\include{incl/main/konklu_perspek/perspektivering}

% Appendicer indsættes inde i en appendices-blok og bliver nummereret med  bogstaver i stedet for tal

\begin{appendices}
\chapter{Fisk}
% God arbejdslyst 
\chapter{Fisk}
% God arbejdslyst xD
\end{appendices}

% Dokumentets 'back matter' er til ekstra ting som f.eks. litteraturlisten.
% Overskrifter bliver ikke nummereret her.

\backmatter

% Automatisk litteraturliste baseret på, hvilke kilder, der er blevet refereret til i løbet af rapporten.

\bibliographystyle{apalike}
\bibliography{
  incl/bib/books,
  incl/bib/articles,
  incl/bib/software
}

\end{document}